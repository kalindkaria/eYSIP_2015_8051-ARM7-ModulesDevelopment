%%%%%%%%%%%%%%%%%%%%%%%%%%%%%%%%%%%%%%%%%%%%%%%%%%%%%%%%%%%%%%%%%%%%%%%%%%%%%
%	E-Yantra, IIT-Bombay

%	Document Author: Bhumika Varshney
%	Date: 30th June,2015

%%%%%%%%%%%%%%%%%%%%%%%%%%%%%%%%%%%%%%%%%%%%%%%%%%%%%%%%%%%%%%%%%%%%%%%%%%%%%

\documentclass[table,10pt,red]{beamer} 
% change the alerted colour to blue
\setbeamercolor{alerted text}{fg=blue}

\usetheme{berlin}
% theme split

\usepackage{beamerthemesplit}
\usepackage{xcolor}
\usepackage{booktabs,array}
\usepackage{listings}
\usepackage{hyperref}
\usepackage{verbatim,moreverb}

\usepackage{colortbl}
\usepackage{multirow}
\usepackage{tikz}
\usetikzlibrary{arrows}


% theme shadow
\usepackage{beamerthemeshadow}

% For including figures
\usepackage{graphicx}

% logo
\logo{\includegraphics[height=1cm]{iitblogo.pdf}}


% sf family, bold font
\sffamily \bfseries
% Beginning of title page
\title
% content inside [] appears at bottom of all page. content inside {} appears on first page as title. double backslash means line change 
[
	Firebird LPC2148 Robotics Research Platform	% bottom
	\hspace{0.5cm}
	\insertframenumber/\inserttotalframenumber
]
{
	Using ADC on Firebird-V Robot
}

\author
[
	www.e-yantra.org
]
{
	e-Yantra Team \\
  Embedded Real-Time Systems Lab\\
  Indian Institute of Technology-Bombay \\
}
\date
{
IIT Bombay \\ {\today}
}
 
 
\begin{document} 

% Slide-1: Title Page
\begin{frame}
	\titlepage
\end{frame} 

% Slide-2: Agenda for Discussion
\section*{Outline}
\begin{frame}
	\frametitle{Agenda for Discussion}
	\tableofcontents
\end{frame} 


\section{Analog to Digital Conversion}
\subsection{Need for ADC}
\begin{frame}
	\frametitle{Need for ADC}
	\begin{enumerate}[$\checkmark$] \pause
		\item <+-|alert@+> IR Proximity sensors 
		\item <+-|alert@+> Sharp IR Range sensors
		\item <+-|alert@+> white line sensors
		\item <+-|alert@+> battery voltage sensing
		\item <+-|alert@+> etc.. 
	\end{enumerate}
\end{frame}

\subsection{ADC of LPC2148}
\begin{frame}
	\frametitle{In-Built ADC of LPC2148} \pause
	\begin{enumerate}[$\checkmark$]
		\item <+-|alert@+> 10-bit Resolution
		\item <+-|alert@+> $>$2.44 $\mu$s Conversion Time
		\item <+-|alert@+> Input multiplexing among 6 pins in ADC0.
		\item <+-|alert@+> Input multiplexing among 8 pins in ADC1.
		\item <+-|alert@+> Power-down mode.
		\item <+-|alert@+> Measurement range 0 V to VREF (typically 3 V).
		\item <+-|alert@+> Burst conversion mode for single or multiple inputs.
		\item <+-|alert@+> Optional conversion on transition on input pin or Timer Match signal.
		\item <+-|alert@+> Global Start command for both converters.
		\item <+-|alert@+> Free Running or Single Conversion Mode
		\item <+-|alert@+> Interrupt on ADC Conversion complete
	\end{enumerate}
\end{frame}	
	
\subsection{ADC Channels}
\begin{frame}[shrink=2]
	\frametitle{ADC Channels}
	\begin{itemize}
		\item Table for ADC Channels \\[20pt]
	\end{itemize}
	\begin{tabular}{!{\vrule width 2pt}c|c|c!{\vrule width 2pt}}
		\noalign{\hrule height 2pt} 
		\textbf{Pin No.}&\textbf{Pin Name}&\textbf{Description}	\\
		\hline
		P0.13&AD1.4&ADC input for Battery Voltage Monitoring	\\
		\hline
		P0.29&AD0.2&ADC input for White Line Sensor 3(Right)	\\
		\hline
		P0.28&AD0.1&ADC input for White Line Sensor 2(Center)	\\
		\hline
		P0.12&AD1.3&ADC input for White Line Sensor 1(Left)	\\
		\hline
		P0.4&AD0.6&ADC input for Sharp IR range sensor 2	\\
		\hline
		P0.6&AD1.0&ADC input for Sharp IR range sensor 3	\\
		\hline
		P0.5&AD0.7&ADC input for Sharp IR range sensor 4	\\
		\hline
		\noalign{\hrule height 2pt}
	\end{tabular} \pause \\[10pt]
\end{frame}


\section{Coding ADC}
\subsection{ADC Initilization}
\begin{frame}
	\frametitle{ADC Initilization}	\pause
	\begin{enumerate}[$\checkmark$]
		\item <+-|alert@+>	To Program ADC, we have to initialize registers before using it.  \\[10pt] 
		The registers are:\pause \\[10pt]
		\color{red}ADxCR \color{black} where x= 0 or 1 \\[10pt] \pause
		\begin{enumerate}
		 	\item <+-|alert@+> \color{red} AD0CR for ADC0 \color{black} - ADC0 Control Register\\[5pt] \pause
	       \item <+-|alert@+> \color{red} AD1CR for ADC1 \color{black} - ADC1 Control Register\\[5pt]	\pause
		\end{enumerate}
		\item <+-|alert@+> Both the Registers are of 32 Bits. \pause
		\\[20pt]
		\color{red}{NOTE: }\color{black}AD0.0 and AD0.5 pins are not available for ADC0 \pause
	\end{enumerate}
\end{frame}


\subsection{ADxCR}
\begin{frame}
		\frametitle{ADxCR- ADCx Control Register} 
		\framesubtitle{This register is Used to control ADC operation} \pause 
			%	\centering
				\begin{tabular}{!{\vrule width 0.8pt}>{\columncolor[gray]{0.9}[0.8\tabcolsep]}c|>{\columncolor[gray]{0.9}[0.8\tabcolsep]}c|>{\columncolor[gray]{0.9}[0.8\tabcolsep]}l|>{\columncolor[gray]{0.9}[0.8\tabcolsep]}c!{\vrule width 0.8pt}}
				\noalign{\hrule height 0.5pt}
				Bit & Symbol & Description & Bit Value  \pause \\  
				\noalign{\hrule height 1pt} 
				\vspace{2pt} 
				7-0 & SEL & Selects pins to be sampled and converted & \pause 00000000 \pause \\
				\vspace{2pt}
				15-8 & CLKDIV & To produce the clock & \pause 0000\color{red}111\color{black}0 \pause \\
				\vspace{2pt}
				16 & BURST & To disable Repeated conversions & \pause 0 \pause \\
				\vspace{2pt}
				19-17 & CLKS & Selects the number of clocks  & \pause 000 \pause\\
				\vspace{2pt}
				20 & - & Reserved & \pause 0 \pause\\
				\vspace{2pt}
				21 & PDN & operational or power-down mode & \pause \color{red} 1\color{black}  \pause\\
				\vspace{2pt}
				23-22 & - & Reserved & \pause 0 \pause\\
				\vspace{2pt}
				26-24 & START & Control start of ADC conversion & \pause 000 \pause\\
				\vspace{2pt}
				27 & EDGE & Rising or falling edge  & \pause 0 \pause\\
				\vspace{2pt}
				31-28 & - & Reserved & \pause 0000 \pause\\
				\noalign{\hrule height 0.5pt}			
			\end{tabular}	\pause \\[10pt]
		\hspace{6cm}ADxCR\hspace{1pt}=\hspace{1pt}\color{red}0x00200E00; \color{black} \pause
\end{frame}

\begin{frame}
	\frametitle{ADxCR} \pause
	\framesubtitle{contd...}  \pause
	\begin{itemize}
		\item Bits 7-0 in ADxCR(SEL)
	\end{itemize}
	\begin{table}
		\footnotesize
	\begin{tabular}{|c|l|}
		\hline  SEL &  Description  \\ 
		\hline \multirow{2}{*}{7} & \color{red}1\color{black}- ADx.7 is sampled and converted \\ 
		                          &\color{red}0\color{black}- ADx.7 is not sampled and converted \\ 
	\hline \multirow{2}{*}{6} & \color{red}1\color{black}- ADx.6 is sampled and converted \\ 
	&\color{red}0\color{black}- ADx.6 is not sampled and converted \\ 
	\hline \multirow{2}{*}{5} & \color{red}1\color{black}- ADx.5 is sampled and converted \\ 
	&\color{red}0\color{black}- ADx.5 is not sampled and converted \\ 
	\hline \multirow{2}{*}{4} & \color{red}1\color{black}- ADx.4 is sampled and converted \\ 
	&\color{red}0\color{black}- ADx.4 is not sampled and converted \\ 
	\hline \multirow{2}{*}{3} & \color{red}1\color{black}- ADx.3 is sampled and converted \\ 
	&\color{red}0\color{black}- ADx.3 is not sampled and converted \\ 
	\hline \multirow{2}{*}{2} & \color{red}1\color{black}- ADx.2 is sampled and converted \\ 
	&\color{red}0\color{black}- ADx.2 is not sampled and converted \\ 
	\hline \multirow{2}{*}{1} & \color{red}1\color{black}- ADx.1 is sampled and converted \\ 
	&\color{red}0\color{black}- ADx.1 is not sampled and converted \\ 
	\hline \multirow{2}{*}{0} & \color{red}1\color{black}- ADx.0 is sampled and converted \\ 
	&\color{red}0\color{black}- ADx.0 is not sampled and converted \\ 
		\hline
	\end{tabular}
	\end{table} 
\end{frame}

\begin{frame}
		\frametitle{ADxCR} 
		\framesubtitle{contd...}  \pause
		\begin{itemize}
			\item Bits 19-17 in ADxCR (CLKS)
		\end{itemize}
				\centering
				\begin{tabular}{|c|l|}
					\hline Bit Value & Function \\ 
					\hline 000 & 11clocks/ 10bits \\ 
					\hline 001 & 10clocks/ 9bits \\ 
					\hline 010 & 9clocks/ 8bits \\ 
					\hline 011 & 8clocks/ 7bits \\ 
					\hline 100 & 7clocks/ 6bits \\ 
					\hline 101 & 6clocks/ 5bits \\ 
					\hline 110 & 5clocks/ 4bits \\ 
					\hline 111 & 4clocks/ 3bits \\ 
					\hline 
				\end{tabular} \\[10pt]
				\pause
\end{frame}

\begin{frame}
	\frametitle{ADxCR} 
	\framesubtitle{contd...}  \pause
	\begin{itemize}
		\item Bits 26-24 in ADxCR(START): 	When the BURST bit is 0, these bits control whether and when an A/D conversion is started,
	\end{itemize}
	\centering
	\begin{tabular}{|c|l|}
		\hline Bit Value & Function \\ 
		\hline 000 & No start \\ 
		\hline 001 & Start Conversion now \\ 
		\hline 010 & Start conversion when the edge occurs on P0.16/MAT0.2 \\ 
		\hline 011 & Start conversion when the edge occurs on P0.22/MAT0.0 \\  
		\hline 100 & Start conversion when the edge occurs on MAT0.1 \\ 
		\hline 101 & Start conversion when the edge occurs on MAT0.3 \\ 
		\hline 110 & Start conversion when the edge occurs on MAT1.0 \\ 
		\hline 111 & Start conversion when the edge occurs on MAT1.1 \\ 
		\hline 
	\end{tabular} \\[10pt]
	\pause
\end{frame}

\subsection{ADxGDR}
\begin{frame}
		\frametitle{ADxGDR - A/D Global Data Register} 
		\framesubtitle{This register contains the ADC's DONE bit and the result of the most recent A/D conversion.} 
			%	\centering
			\begin{table}
				\small
			\begin{tabular}{|c|c|l|}
				\hline Bit & Symbol & Description \\ 
				\hline 5-0 & - & Reserved \\ 
				\hline \multirow{2}{*}{15-6}  & RESULT & When DONE is 1, this \\ 
											  &  & field contains ADC converted data \\ 
				\hline 23-16 & - & Reserved \\ 
				\hline 26-24 & CHN & Channel Number \\ 
				\hline 29-27 & - & Reserved \\ 
				\hline \multirow{2}{*}{30} & OVERUN & 1 if the results of one or more conversions were lost before\\
				                           &        & the conversion that produced the result in the RESULT bits. \\
				\hline 31 & DONE & This bit is set to 1 when an A/D conversion completes. \\ 
				\hline 
			\end{tabular}
			\end{table}
\end{frame}

\subsection{Program}
\begin{frame}[shrink = 4,fragile]
	\frametitle{Syntax for C-Program} \pause
	\framesubtitle{ADC Initialization}
		\begin{block}<1->{Init\_ADC\_Pin}	\pause
		\begin{semiverbatim}
				\scriptsize{
				void Init\_ADC\_Pin (void) \color{green} //Configure ADC Ports  \color{black}
				\{
					 
					PINSEL0= 0x0F003F00;  \color{red}//Set pins P0.4, P0.5, P0.6, P0.12, P0.13 as ADC pins\color{black}
					PINSEL1= 0x05000000;  \color{red}//Set pins P0.28 and P0.29 as ADC pins\color{black}
					 
				\} }
			\end{semiverbatim}
		\end{block} \pause
	
	\begin{block}<1->{ADC Initialization}	\pause
		\begin{semiverbatim}
				\scriptsize{
				void Init\_ADC()			\color{green} //Set Register Values for starting ADC  \color{black}
				\{
					 
  					AD0CR =
  					AD1CR =  
				\} }
			\end{semiverbatim}
		\end{block} \pause
\end{frame}

\begin{frame}[shrink = 2,fragile]
	\frametitle{Syntax for C-Program} \pause
	\framesubtitle{Program}
		\begin{block}<1->{Main Program}	\pause
		\begin{semiverbatim}
				\scriptsize{
				int main(void)
				\{
			\ \		Init\_Peripherals();
			\ \		LCD\_Init();
			\ \		while(1)
			\ \		\{
			\ \ \ \		LCD_Print(2,1,AD1_Conversion(3),3);   //whiteline Left 
			\ \ \ \ LCD_Print(2,7,AD0_Conversion(1),3);   //whiteline Center	 
			\ \ \ \ LCD_Print(2,13,AD0_Conversion(2),3);	//whiteline Right
			\ \		\}
				\}
 }
			\end{semiverbatim}
		\end{block} \pause
\end{frame}

\begin{frame}[shrink = 5,fragile]
	\frametitle{Syntax for C-Program} \pause
	\framesubtitle{Program}
		\begin{block}<1->{AD0 Conversion Function}	\pause
		\begin{semiverbatim}
				\scriptsize{
				unsigned char AD0_Conversion(unsigned char Ch)
				\{
					unsigned int Temp;
					if(channel!=0)
					\{
					\ \ \ \	AD0CR = (AD0CR & 0xFFFFFF00) | (1<<channel);
					\}
					else
					\{
					\ \ \ \	AD0CR = (AD0CR & 0xFFFFFF00) | 0x01;
					\}
					AD0CR|=(1 << 24);
					while((AD0GDR & 0x80000000)==0);
					Temp = AD0GDR;						
					Temp = (Temp>>8) & 0xFF;
					return Temp;
				\}
 }
			\end{semiverbatim}
		\end{block}
\end{frame}

\begin{frame}[shrink = 5,fragile]
	\frametitle{Syntax for C-Program} \pause
	\framesubtitle{Program}
	\begin{block}<1->{AD1 Conversion Function}	\pause
		\begin{semiverbatim}
			\scriptsize{
				unsigned char AD1_Conversion(unsigned char Ch)
				\{
				unsigned int Temp;
				if(channel!=0)
				\{
				\ \ \ \	AD1CR = (AD1CR & 0xFFFFFF00) | (1<<channel);
				\}
				else
				\{
				\ \ \ \	AD1CR = (AD1CR & 0xFFFFFF00) | 0x01;
				\}
				AD1CR|=(1 << 24);
				while((AD1GDR & 0x80000000)==0);
				Temp = AD1GDR;						
				Temp = (Temp>>8) & 0xFF;
				return Temp;
				\}
			}
		\end{semiverbatim}
	\end{block}
\end{frame}


\begin{frame}
\hskip4cm
\textbf{\LARGE Thank You!} \\[20pt]
\hskip3cm
\scriptsize Post your queries on: 
\hyperref[www.e-yantra.org]{\color{blue} http://qa.e-yantra.org/ \color{black}} 
\end{frame}

\end{document} 